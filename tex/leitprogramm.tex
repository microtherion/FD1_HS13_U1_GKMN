\documentclass[12pt,a4paper]{report}

\usepackage{times}
\usepackage[german]{babel}
\usepackage{graphics}
\usepackage{color}
\usepackage{ntheorem}
\theoremstyle{break}
\theorembodyfont{\normalfont}
\theoremprework{\bigskip\hrule\leavevmode}
\theorempostwork{\nopagebreak\hrule\leavevmode}
\newtheorem{exercise}{Aufgabe}[section]

\title{Minimale Spannb\"{a}ume}
\author{Gabriel Katz\\ Matthias Neeracher}

\begin{document}
\maketitle
\tableofcontents
\chapter{Einleitung}

\section{Wie kann hier gearbeitet werden?}

Dieser Text wird Sie mit einem wichtigen algorithmischen Problem
bekannt machen: dem Finden eines minimalen Spannbaums. Im n\"{a}chsten
Unterkapitel finden Sie eine Einf\"{u}hrung in das Thema, doch bevor
Sie loslegen, soll Ihnen hier noch kurz der Aufbau des Textes und die
Arbeitsweise damit erl\"{a}utert werden.

Die folgenden Kapitel sind zur selbstst\"{a}ndigen Bearbeitung
gedacht. Sie werden zuerst kurz Ihre Kenntnisse von ungerichteten
Graphen nochmals auffrischen, lernen dann die Theorie von B\"{a}umen
und gerichteten Graphen kennen, und lernen dann zwei verschiedene
Algorithmen, um einen \emph{minimalen Spannbaum} zu finden.
 
Wenn Sie einige dieser Begriffe jetzt noch nicht verstehen, macht das
nichts. Mitbringen sollten Sie aber die folgenden Kenntnisse:

\begin{itemize}
\item Sie wissen, was ein Algorithmus ist.
\item Sie kennen die Grundbegriffe ungerichteter Graphen (wir werden
  diese allerdings in Kapitel \ref{graphs} kurz repetieren.
\end{itemize}

Zur behandelten Theorie finden Sie immer auch Aufgaben, anhand derer
Sie das Gelernte pr\"{u}fen k\"{o}nnen. Die L\"{o}sungen dieser Aufgaben stehen
jeweils im zweitletzten Unterkapitel f\"{u}r jedes Thema. Das letzte
Unterkapitel ist dann der Kapiteltest, den Sie bearbeiten und mit
Ihrer Lehrperson besprechen sollten.

\section{Worum geht es hier?}

In den fr\"uhen 20er Jahren besch\"aftigte sich der tschechische
Mathematiker Otakar Bor\r{u}vka mit dem Problem, ein Gebiet
m\"{o}glichst effizient mit Elektrizit\"{a}t zu erschliessen. Sein
Ansatz war, das Problem als Graphen abzubilden, in dem die
anzuschliessenden Punkte als Knoten, und die Distanzen zwischen ihnen
als Kanten repr\"{a}sentiert wurden:

Das Elektrifizierungsproblem besteht somit darin, alle Knoten so zu
verbinden, dass die Gesamtl\"{a}nge der Kanten so kurz wie m\"{o}glich bleibt:  

Bor\r{u}vka's L\"{o}sung, die er 1926 publizierte, gilt als der erste
Algorithmus zur Konstruktion eines minimalen Spannbaums. Weil sowohl
die Problemstellung als auch die verwendeten Algorithmen interessant
und einigermassen leicht verst\"{a}ndlich sind, werden wir uns jetzt diesem Problem widmen.

\chapter{Graphen}
\label{graphs}

\section{Rekapitulation: Was ist ein Graph?}

Ein \textbf{Graph} $G(V,E)$ besteht aus einer Menge von
\textbf{Knoten} $V$ und einer Menge von Kante $E$. Eine \textbf{Kante}
besteht aus einem Paar von Knoten, den \textbf{Endknoten} der
Kante. Wenn diese Paare geordnet sind, ist der Graph
\textbf{gerichtet}; die Kanten verbinden die Knoten nur in eine
Richtung. 

In diesem Text befassen wir uns aber nur mit
\textbf{ungerichteten} Graphen, in welchen die Knotenpaare ungeordnet
und somit die Knoten symmetrisch verbunden sind.  Des weiteren
beschr\"{a}nken wir uns auf \textbf{schlichte} Graphen, in denen keine
Kanten einen Knoten mit sich selbst verbinden, und zwischen zwei
Knoten nicht mehr als eine Kante existiert.

\begin{exercise}
Welche dieser vier Graphen sind gerichtet, welche ungerichtet? Welche
sind schlicht?

\scalebox{0.4}{\includegraphics{../graph/img/UngerichtetUnschlicht.pdf}}
\scalebox{0.4}{\includegraphics{../graph/img/Gerichtet.pdf}}
\scalebox{0.4}{\includegraphics{../graph/img/Ungerichtet.pdf}}
\scalebox{0.4}{\includegraphics{../graph/img/GerichtetUnschlicht.pdf}}
\end{exercise}

\section{Beschreibung von Graphen}
\label{Beschreibung}
\noindent Ein Graph $G(V,E)$ kann auf verschiedene Arten beschrieben werden:

\begin{description}
\item[Grafisch] indem die Knoten und Kanten aufgezeichnet werden:

\scalebox{0.5}{\includegraphics{../graph/img/Ungerichtet.pdf}}

gerade bei kleineren Graphen ist diese Darstellung f\"{u}r Menschen am
Einfachsten zu verstehen, aber sie ist z.B. f\"{u}r
Computerverarbeitung nicht besonders gut geeignet.
\item[Mengentheoretisch] indem die Knoten- und die Kantenmenge
  beschrieben werden:

\begin{displaymath}
V = \{\mathsf{A,B,C,D,E}\}\hspace{2cm}E = \{\mathsf{(A,B), (B,C),
  (C,D), (C,E), (B,E)}\}
\end{displaymath}

\item[\textnormal{als} Adjazenzmatrix] indem die Kanten als $n\times{n}$ Matrix $A(G)$
  dargestellt werden, wo der Eintrag $a_{ij}$ gleich $1$ ist, wenn eine Kante
  von Knoten $i$ nach Knoten $j$ existiert, und sonst gleich $0$.

\begin{displaymath}
A(G) = \left( 
\begin{array}{ccccc}
0 & 1 & 0 & 0 & 0 \\
1 & 0 & 1 & 0 & 1 \\
0 & 1 & 0 & 1 & 1 \\
0 & 0 & 1 & 0 & 0 \\
0 & 1 & 1 & 0 & 0 
\end{array}
\right)
\end{displaymath}

Wie zu beobachten ist, ist die Adjazenzmatrix bei einem ungerichteten
Graphen immer symmetrisch.
\end{description}

\begin{exercise}
\begin{itemize}
\item Zeichnen Sie einen ungerichteten Graphen $G(V,E)$ mit Knoten 
$V = \{\mathsf{A,B,C,D,E}\}$ und Kanten 
$E = \{\mathsf{(A, B), (A, D), (C, D), (A, E)}\}$.
\item Konstruieren sie die Adjazenzmatrix dieses Graphen.
\end{itemize}
\end{exercise}

\section{Zusammenh\"{a}ngende Graphen}

Wenn in einem Graphen $G(V,E)$ eine Folge von Knoten $v_0, v_1,
\ldots, v_n \in V$ existiert, die von Kanten $e_i = (v_i, v_i+1) \in
E$ verbunden werden, dann existiert ein \textbf{Weg} von $v_0$ nach
$v_n$. Falls solch ein Weg f\"{u}r \emph{jedes} Knotenpaar $v_0, v_n
\in V$ existiert, nennen wir den Graphen \textbf{zusammenh\"{a}ngend}.

\begin{exercise}
\textcolor{red}{Irgendwelche Ideen?}
\end{exercise}

\section{Untergraphen}

Ein Graph $U(W,F)$ ist ein \textbf{Untergraph} eines Graphen $G(V,E)$
wenn seine Knotenmenge $W$ eine Untermenge der Knotenmenge $V$ ist und
seine Kantenmenge $F$ eine Untermenge der Kantenmenge $E$.

\begin{exercise}
Welche der folgenden drei Graphen sind Untergraphen des Graphen in
Abschnitt \ref{Beschreibung} ?

\scalebox{0.4}{\includegraphics{../graph/img/Sub1.pdf}}
\scalebox{0.4}{\includegraphics{../graph/img/Sub2.pdf}}
\scalebox{0.4}{\includegraphics{../graph/img/Sub3.pdf}}

\end{exercise}

\newpage
\section{Gewichtete Graphen}

In vielen praktischen Anwendungen sind die Knoten und/oder Kanten
eines Graphen mit \textbf{Gewichten} versehen. Die Kantengewichte k\"{o}nnen
z.B. Distanz oder Flugkosten darstellen, die Knotengewichte
z.B. Hotelkosten.

\begin{exercise}
Der folgende Graph zeigt Flug- und Hotelkosten, sowie Reisedistanzen
f\"{u}r eine Reise.

\scalebox{0.5}{\includegraphics{../graph/img/Travel.pdf}}

\begin{enumerate}
\item Was ist die \emph{k\"{u}rzeste} Route zwischen Z\"{u}rich und
    Sydney?
\item Was ist die \emph{billigste} Route?
\end{enumerate}
\end{exercise}

 In der weiteren Diskussion werden wir hier nur noch an \textbf{Kantengewichteten} Graphen interessiert sein.

\chapter{B\"{a}ume}
\chapter{Algorithmen}
\section{Der Algorithmus von Kruskal}
\section{Der Algorithmus von Prim}
\end{document}
