\documentclass[12pt,a4paper]{report}

\usepackage{times}
\usepackage[german]{babel}
\usepackage{ntheorem}
\usepackage{graphics}
\theoremstyle{break}
\theorembodyfont{\normalfont}
\theoremprework{\bigskip\hrule\leavevmode}
\theorempostwork{\nopagebreak\hrule\leavevmode}
\newtheorem{exercise}{Aufgabe}[section]

\title{Minimale Spannb\"{a}ume}
\author{Gabriel Katz\\ Matthias Neeracher}

\begin{document}
\maketitle
\tableofcontents
\chapter{Einleitung}

\section{Wie kann hier gearbeitet werden?}

Dieser Text wird Sie mit einem wichtigen algorithmischen Problem
bekannt machen: dem Finden eines minimalen Spannbaums. Im n\"{a}chsten
Unterkapitel finden Sie eine Einf\"{u}hrung in das Thema, doch bevor
Sie loslegen, soll Ihnen hier noch kurz der Aufbau des Textes und die
Arbeitsweise damit erl\"{a}utert werden.

Die folgenden Kapitel sind zur selbstst\"{a}ndigen Bearbeitung
gedacht. Sie werden zuerst kurz Ihre Kenntnisse von ungerichteten
Graphen nochmals auffrischen, lernen dann die Theorie von B\"{a}umen
und gerichteten Graphen kennen, und lernen dann zwei verschiedene
Algorithmen, um einen \emph{minimalen Spannbaum} zu finden.
 
Wenn Sie einige dieser Begriffe jetzt noch nicht verstehen, macht das
nichts. Mitbringen sollten Sie aber die folgenden Kenntnisse:

\begin{itemize}
\item Sie wissen, was ein Algorithmus ist.
\item Sie kennen die Grundbegriffe ungerichteter Graphen (wir werden
  diese allerdings in Kapitel \ref{graphs} kurz repetieren.
\end{itemize}

Zur behandelten Theorie finden Sie immer auch Aufgaben, anhand derer
Sie das Gelernte pr\"{u}fen k\"{o}nnen. Die L\"{o}sungen dieser Aufgaben stehen
jeweils im zweitletzten Unterkapitel f\"{u}r jedes Thema. Das letzte
Unterkapitel ist dann der Kapiteltest, den Sie bearbeiten und mit
Ihrer Lehrperson besprechen sollten.

\section{Worum geht es hier?}

In den fr\"uhen 20er Jahren besch\"aftigte sich der tschechische
Mathematiker Otakar Bor\r{u}vka mit dem Problem, ein Gebiet
m\"{o}glichst effizient mit Elektrizit\"{a}t zu erschliessen. Sein
Ansatz war, das Problem als Graphen abzubilden, in dem die
anzuschliessenden Punkte als Knoten, und die Distanzen zwischen ihnen
als Kanten repr\"{a}sentiert wurden:

Das Elektrifizierungsproblem besteht somit darin, alle Knoten so zu
verbinden, dass die Gesamtl\"{a}nge der Kanten so kurz wie m\"{o}glich bleibt:  

Bor\r{u}vka's L\"{o}sung, die er 1926 publizierte, gilt als der erste
Algorithmus zur Konstruktion eines minimalen Spannbaums. Weil sowohl
die Problemstellung als auch die verwendeten Algorithmen einigermassen
leicht verst\"{a}ndlich sind, werden wir uns jetzt diesem Problem widmen.

\chapter{Graphen}
\label{graphs}

\section{Rekapitulation: Was ist ein Graph?}

Ein \textbf{Graph} $G(V,E)$ besteht aus einer Menge von
\textbf{Knoten} $V$ und einer Menge von Kante $E$. Eine \textbf{Kante}
besteht aus einem Paar von Knoten, den \textbf{Endknoten} der
Kante. Wenn diese Paare geordnet sind, ist der Graph
\textbf{gerichtet}; die Kanten verbinden die Knoten nur in eine
Richtung. 

In diesem Text befassen wir uns aber nur mit
\textbf{ungerichteten} Graphen, in welchen die Knotenpaare ungeordnet
und somit die Knoten symmetrisch verbunden sind.  Des weiteren
beschr\"{a}nken wir uns auf \textbf{schlichte} Graphen, in denen keine
Kanten einen Knoten mit sich selbst verbinden, und zwischen zwei
Knoten nicht mehr als eine Kante existiert.

\begin{exercise}
Welche dieser vier Graphen sind gerichtet, welche ungerichtet? Welche
sind schlicht?

\resizebox{3.5cm}{!}{\includegraphics{../graph/img/UngerichtetUnschlicht.pdf}}
\resizebox{3.5cm}{!}{\includegraphics{../graph/img/Gerichtet.pdf}}
\resizebox{3.5cm}{!}{\includegraphics{../graph/img/Ungerichtet.pdf}}
\resizebox{3.5cm}{!}{\includegraphics{../graph/img/GerichtetUnschlicht.pdf}}
\end{exercise}

\begin{exercise}
Zeichnen Sie einen ungerichteten Graphen $G(V,E)$ mit Knoten 
$V = \{\mathsf{A,B,C,D,E}\}$ und Kanten 
$E = \{\mathsf{(A, B), (A, D), (C, D), (A, E)}\}$.
\end{exercise}
 
\chapter{B\"{a}ume}
\chapter{Algorithmen}
\section{Der Algorithmus von Kruskal}
\section{Der Algorithmus von Prim}
\end{document}
