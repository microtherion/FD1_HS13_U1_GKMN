\documentclass[12pt,a4paper]{report}

\setlength{\topmargin}{0 cm}
\usepackage{times}
\usepackage{float}
\usepackage{rotating}
\usepackage[german]{babel}
\usepackage{graphics}
\graphicspath{ {../img/} {../graph/img/} }
\usepackage[table,dvipsnames]{xcolor}
\usepackage{caption}
\usepackage{subcaption}
\usepackage{ntheorem}
\usepackage{pifont}
\theoremstyle{break}
\theorembodyfont{\normalfont}
\theoremprework{\bigskip\hrule\leavevmode\nopagebreak}
\theorempostwork{\nopagebreak\hrule\leavevmode}
\newtheorem{exercise}{Aufgabe}[chapter]
\theoremstyle{plain}
\theoremprework{\bigskip\hrule\leavevmode\nopagebreak}
\theorempostwork{\nopagebreak\hrule\leavevmode}
\newtheorem{proof}{Satz}[chapter]
\renewcommand{\labelenumii}{\arabic{enumi}.\arabic{enumii}.}
\newcommand{\algostep}[2]{\noindent\parbox{4cm}{\scalebox{0.5}{\includegraphics{#1}}}
  \hfill
  \parbox{7cm}{#2}
  \vskip -5mm
}
\newcommand{\ucl}{\ding{56}}
\newcommand{\ign}{\rowcolor[gray]{0.5}}
\newcommand{\usd}[1]{{\cellcolor{SkyBlue}#1}}
\newcommand{\sel}[1]{{\cellcolor{LimeGreen}#1}}
\newcommand{\byp}[1]{{\cellcolor{LimeGreen!10}#1}}
\newcommand{\matrixstep}[2]{\noindent\parbox{6cm}{\scriptsize #1}
  \hfill
  \parbox{7cm}{\small #2}
  \vskip 2mm
}

\title{Minimale Spannb\"{a}ume}
\author{Gabriel Katz\\ Matthias Neeracher}

\begin{document}
\maketitle
\tableofcontents
\chapter{Einleitung}

\section{Worum geht es hier?}

In den fr\"uhen 20er Jahren besch\"aftigte sich der 
Mathematiker Otakar Bor\r{u}vka mit dem Problem, ein Gebiet (40
St\"{a}dte in S\"{u}d-M\"{a}hren) m\"{o}glichst effizient mit 
Elektrizit\"{a}t zu erschliessen.

\begin{figure}[h!]
\resizebox{!}{4cm}{\includegraphics{BoruvkaPoints.pdf}}
\resizebox{!}{4cm}{\includegraphics{BoruvkaTree.pdf}}
\caption{Elektrische Erschliessung von S\"{u}d-M\"{a}hren.\protect\footnotemark}
\end{figure}

\footnotetext{ 
Otakar  Bor\r{u}vka, \emph{O jist\'{e}m probl\'{e}mu minim\'{a}ln\'{i}m} (\"{U}ber ein
  gewisses Minimalisierungsproblem). \\
Zitiert aus: J. Ne\v{s}et\v{r}il,
  E. Milkov\'{a}, H. Ne\v{s}et\v{r}ilov\'{a}, \emph{Otakar Bor\r{u}vka
    on minimum spanning tree problem: translation of both the 1926
    papers, comments, history.}, Discrete Mathematics 233 (2001)}
Sein Ansatz war, das Problem als \textbf{Graphen} abzubilden, in dem die
anzuschliessenden St\"{a}dte als \textbf{Knoten}, und die Distanzen zwischen ihnen
als \textbf{Kanten} repr\"{a}sentiert wurden. Das Elektrifizierungsproblem
besteht somit darin, alle Knoten so zu verbinden, dass die
Gesamtl\"{a}nge der Kanten so kurz wie m\"{o}glich bleibt. Eine solche
Verbindung wird \textbf{minimal aufspannender Baum} oder kurz
\textbf{minimaler Spannbaum} genannt.

Bor\r{u}vka's L\"{o}sung, die er 1926 publizierte, gilt als der erste
Algorithmus zur Konstruktion eines minimalen Spannbaums, und als einer
der ersten \textbf{Optimierungs-Algorithmen}, d.h. ein Algorithmus,
der eine L\"{o}sung findet, bei der ein bestimmter Wert m\"{o}glichst
klein\footnote{Oder bei anderen Problemen m\"{o}glichst gross} werden soll. Weil sowohl
die Problemstellung als auch die verwendeten Algorithmen interessant
und einigermassen leicht verst\"{a}ndlich sind, werden wir uns jetzt diesem Problem widmen.
\section{Wie kann hier gearbeitet werden?}

Dieser Text wird Sie mit einem wichtigen algorithmischen Problem
bekannt machen: dem Finden eines minimalen Spannbaums. Im n\"{a}chsten
Abschnitt finden Sie eine Einf\"{u}hrung in das Thema, doch bevor
Sie loslegen, soll Ihnen hier noch kurz der Aufbau des Textes und die
Arbeitsweise damit erl\"{a}utert werden.

Die folgenden Kapitel sind zur selbstst\"{a}ndigen Bearbeitung
gedacht. Sie werden zuerst kurz Ihre Kenntnisse von ungerichteten
Graphen nochmals auffrischen, lernen dann die Theorie von B\"{a}umen
und gewichteten Graphen kennen, und lernen dann zwei verschiedene
Algorithmen, um einen \emph{minimalen Spannbaum} zu finden.
 
Wenn Sie einige dieser Begriffe jetzt noch nicht verstehen, macht das
nichts. Mitbringen sollten Sie aber die folgenden Kenntnisse:

\begin{itemize}
\item Sie wissen, was ein Algorithmus ist.
\item Sie kennen die Grundbegriffe ungerichteter Graphen (wir werden
  diese allerdings in Kapitel~\ref{graphs} kurz repetieren).
\end{itemize}

Zur behandelten Theorie finden Sie immer auch Aufgaben, anhand derer
Sie das Gelernte pr\"{u}fen k\"{o}nnen. Die L\"{o}sungen dieser Aufgaben stehen
jeweils im zweitletzten Unterkapitel f\"{u}r jedes Thema. Das letzte
Unterkapitel ist dann der Kapiteltest, den Sie bearbeiten und mit
Ihrer Lehrperson besprechen sollten.

\newpage
\chapter{Graphen}
\label{graphs}
In diesem Kapitel frischen wir unsere Kenntnisse zur Graphentheorie
auf. Zuerst repetieren wir, was ein Graph \"{u}berhaupt ist und wie er
dargestellt werden kann. Danach betrachten wir zwei Konzepte,
n\"{a}mlich zusammenh\"{a}ngende Graphen und Teilgraphen, welche
f\"{u}r die n\"{a}chsten Kapitel wichtig sind. Am Ende des Kapitels
behandeln wir noch die gewichteten Graphen. Wenn Sie sich bereits
in all diesen Themen sattelfest f\"{u}hlen, k\"{o}nnen Sie direkt den
Kapiteltest in Abschnitt~\ref{graphchaptest} in Angriff nehmen.

\section{Rekapitulation: Was ist ein Graph?}

Ein \textbf{Graph} $G(V,E)$ besteht aus einer Menge von
\textbf{Knoten} $V$ und einer Menge von Kante $E$. Eine \textbf{Kante}
besteht aus einem Paar von Knoten, den \textbf{Endknoten} der
Kante. Wenn diese Paare geordnet sind, ist der Graph
\textbf{gerichtet}; die Kanten verbinden die Knoten nur in eine
Richtung. 

In diesem Text befassen wir uns aber nur mit
\textbf{ungerichteten} Graphen, in welchen die Knotenpaare ungeordnet
und somit die Knoten symmetrisch verbunden sind.  Des weiteren
beschr\"{a}nken wir uns auf \textbf{einfache} Graphen, in denen keine
Kanten einen Knoten mit sich selbst verbinden, und zwischen zwei
Knoten nicht mehr als eine Kante existiert.

\begin{exercise}\label{exgerichtet}
Welche dieser vier Graphen sind gerichtet, welche ungerichtet? Welche
sind einfach?
\begin{figure}[H]
\begin{subfigure}[b]{0.23\textwidth}
\scalebox{0.4}{\includegraphics{UngerichtetUnschlicht.pdf}}
\caption{}
\end{subfigure}
\begin{subfigure}[b]{0.23\textwidth}
\scalebox{0.4}{\includegraphics{Gerichtet.pdf}}
\caption{}
\end{subfigure}
\begin{subfigure}[b]{0.23\textwidth}
\scalebox{0.4}{\includegraphics{Ungerichtet.pdf}}
\caption{}
\end{subfigure}
\begin{subfigure}[b]{0.23\textwidth}
\scalebox{0.4}{\includegraphics{GerichtetUnschlicht.pdf}}
\caption{}
\end{subfigure}
\caption{Welche der Graphen sind gerichtet? Welche einfach?}
\end{figure}
\end{exercise}

\section{Beschreibung von Graphen}
\label{Beschreibung}
\noindent Ein Graph $G(V,E)$ kann auf verschiedene Arten beschrieben werden:

\begin{description}
\item[Grafisch] indem die Knoten und Kanten aufgezeichnet werden:

\begin{figure}[H]
\centerline{
\scalebox{0.5}{\includegraphics{Ungerichtet.pdf}}
}
\caption{Grafische Darstellung eines Graphen}
\label{fig:GrafischeDarstellung}
\end{figure}
gerade bei kleineren Graphen ist diese Darstellung f\"{u}r Menschen am
Einfachsten zu verstehen, aber sie ist z.B. f\"{u}r
Computerverarbeitung nicht besonders gut geeignet.
\item[Mengentheoretisch] indem die Knoten- und die Kantenmenge
  beschrieben werden:

\begin{displaymath}
V = \{\mathsf{A,B,C,D,E}\}\hspace{2cm}E = \{\mathsf{\{A,B\}, \{B,C\},
  \{C,D\}, \{C,E\}, \{B,E\}}\}
\end{displaymath}

\item[\textnormal{als} Adjazenzmatrix] indem die Kanten als $n\times{n}$ Matrix $A(G)$
  dargestellt werden, wo der Eintrag $a_{ij}$ gleich $1$ ist, wenn eine Kante
  von Knoten $i$ nach Knoten $j$ existiert, und sonst gleich $0$.

\begin{displaymath}
A(G) = \left( 
\begin{array}{ccccc}
0 & 1 & 0 & 0 & 0 \\
1 & 0 & 1 & 0 & 1 \\
0 & 1 & 0 & 1 & 1 \\
0 & 0 & 1 & 0 & 0 \\
0 & 1 & 1 & 0 & 0 
\end{array}
\right)
\end{displaymath}

Wie zu beobachten ist, ist die Adjazenzmatrix bei einem ungerichteten
Graphen immer symmetrisch.
\end{description}

\begin{exercise}\label{exrep}
\begin{itemize}
\item Zeichnen Sie einen ungerichteten Graphen $G(V,E)$ mit Knoten 
$V = \{\mathsf{A,B,C,D,E}\}$ und Kanten 
$E = \{\mathsf{\{A, B\}, \{A, D\}, \{C, D\}, \{A, E\}}\}$.
\item Konstruieren Sie die Adjazenzmatrix dieses Graphen.
\end{itemize}
\end{exercise}

\section{Zusammenh\"{a}ngende Graphen}

Wenn in einem Graphen $G(V,E)$ eine Folge von Knoten $(v_0, v_1,
\ldots, v_n)$, $v_i \in V$ existiert, die von Kanten $e_i = \{v_i, v_{i+1}\} \in
E$ verbunden werden, dann existiert ein \textbf{Weg} von $v_0$ nach
$v_n$. 

Falls ein Weg f\"{u}r \emph{jedes} Knotenpaar $v_0, v_n
\in V$ existiert, nennen wir den Graphen \textbf{zusammenh\"{a}ngend}.

Ob ein Graph zusammenh\"{a}ngend ist, l\"{a}sst sich intuitiv recht
einfach in seiner grafischen Darstellung verstehen: Ein Graph
ist zusammenh\"{a}ngend, wenn alle Knoten miteinander verbunden sind. Sind
die Knoten nicht alle miteinander verbunden, so heissen die verbundenen Teile des 
Graphen \textbf{Komponenten}.

Ein Weg $v_0,v_1,\ldots,v_n,{v_0}$, der von einem Knoten
$v_0$ wieder auf diesen selbst zur\"{u}ckf\"{u}hrt, heisst
\textbf{geschlossen}.

\begin{exercise}\label{exverbund}
Welcher der folgenden Graphen ist zusammenh\"{a}ngend? Markiere bei den nicht zusammenh\"{a}ngenden Graphen die
gr\"{o}sste Komponente.

\begin{figure}[H]
\begin{subfigure}[b]{0.3\textwidth}
\scalebox{0.4}{\includegraphics{ConnectedComponents1.pdf}}
\caption{}
\end{subfigure}
\begin{subfigure}[b]{0.3\textwidth}
\scalebox{0.4}{\includegraphics{ConnectedComponents2.pdf}}
\caption{}
\end{subfigure}
\begin{subfigure}[b]{0.3\textwidth}
\scalebox{0.4}{\includegraphics{ConnectedComponents3.pdf}}
\caption{}
\end{subfigure}

\caption{Welcher Graph ist zusammenh\"{a}ngend?}
\end{figure}

\end{exercise}

\newpage
\section{Teilgraphen}

Ein Graph $U(W,F)$ ist ein \textbf{Teilgraph} eines Graphen $G(V,E)$
wenn seine Knotenmenge $W$ eine Teilmenge der Knotenmenge $V$ ist und
seine Kantenmenge $F$ eine Teilmenge der Kantenmenge $E$.
Ein Teilgraph $U(W,F)$ eines Graphen $G$ ist \textbf{aufspannend}, 
wenn $W=V$. Ein aufspannender Teilgraph muss weder zusammenh\"{a}ngend sein, noch muss jeder Knoten mit einer Kante verbunden sein.
\begin{exercise}\label{exunter}
Welche der folgenden drei Graphen sind Teilgraphen des Graphen aus
Abbildung~\ref{fig:GrafischeDarstellung}?

\begin{figure}[H]
\begin{subfigure}[b]{0.3\textwidth}
\scalebox{0.4}{\includegraphics{Sub1.pdf}}
\caption{}
\end{subfigure}
\begin{subfigure}[b]{0.3\textwidth}
\scalebox{0.4}{\includegraphics{Sub2.pdf}}
\caption{}
\end{subfigure}
\begin{subfigure}[b]{0.3\textwidth}
\scalebox{0.4}{\includegraphics{Sub3.pdf}}
\caption{}
\end{subfigure}

\caption{Welche Graphen sind Teilgraphen?}
\end{figure}

\end{exercise}
\begin{exercise}\label{excompletesubgraph}
Ein \emph{vollst\"{a}ndiger} Graph $K_n$ ist ein Graph mit $n$ Knoten, in welchem jedes Knotenpaar mit einer Kante verbunden ist.
\[ 
K_n(V = \{v_1, v_2, \ldots, v_n\}, E = \{\{v_a, v_b\} \mid v_a, v_b \in V,
v_a \neq v_b\})
\] 
\begin{itemize}
\item Finde eine Formel f\"ur die Anzahl Kanten in $K_n$
\item \textbf{(FAKULTATIV)} Wie viele aufspannende Teilgraphen hat
  $K_2$? Wieviele $K_3$? Finde eine Formel, um die Anzahl
  aufspannender Teilgraphen von $K_n$ zu berechnen.
\end{itemize}
\end{exercise}

\newpage
\section{Gewichtete Graphen}
\label{gewichtet}

In vielen praktischen Anwendungen sind die Knoten und/oder Kanten
eines Graphen mit \textbf{Gewichten} versehen. Die Kantengewichte k\"{o}nnen
z.B. Distanzen oder Flugkosten darstellen, die Knotengewichte
z.B. Hotelkosten.

\begin{exercise}\label{extravel}
Der folgende Graph zeigt Flug- und Hotelkosten, sowie Reisedistanzen
f\"{u}r eine Reise. Wir nehmen an, dass der Reisende bei jedem Zwischenhalt eine Nacht in einem Hotel \"{u}bernachten muss.
Bei Start- und Endpunkt der Reise, also, Z�rich und Sidney, ist keine \"{U}bernachtung n\"{o}tig. Daher sind im Graph auch 
keine Kosten aufgef\"{u}hrt.
\begin{figure}[H]
\centerline{
\scalebox{0.5}{\includegraphics{Travel.pdf}}
}
\caption{Flug- und Hotelkosten und Reisedistanzen}
\end{figure}
\begin{enumerate}
\item Was ist die \emph{k\"{u}rzeste} Route zwischen Z\"{u}rich und
    Sydney?
\item Was ist die \emph{billigste} Route, wenn man die Hotelkosten bei jeder Zwischenlandung ber�cksichtigt?
\end{enumerate}
\end{exercise}

In der weiteren Diskussion werden wir hier nur noch an
\textbf{Kantengewichteten} Graphen interessiert sein. 

\newpage
\section{L\"{o}sungen}

\begin{description}
\item[Aufgabe~\ref{exgerichtet}] \hfill \\[0cm]
\begin{enumerate}
\item Ungerichtet, nicht schlicht weil $C$
  mit sich selbst verbunden ist. 
\item Gerichtet und schlicht. 
\item Ungerichtet
  und schlicht. 
\item Gerichtet, nicht schlicht weil zwischen $B$ und $E$
  zwei Kanten existieren.
\end{enumerate}
\item[Aufgabe~\ref{exrep}] \hfill \\[0cm]
\algostep{Exrep.pdf}{
\begin{displaymath}
\left( 
\begin{array}{ccccc}
0 & 1 & 0 & 1 & 1 \\
1 & 0 & 0 & 0 & 0 \\
0 & 0 & 0 & 1 & 0 \\
1 & 0 & 1 & 0 & 0 \\
1 & 0 & 0 & 0 & 0 
\end{array}
\right)
\end{displaymath}}
\item[Aufgabe~\ref{exverbund}] Graph (a) ist nicht verbunden. Die 
gr\"{o}sste Komponente besteht aus den  Knoten $A$, $B$ und $C$.
Graph (b) ist zusammenh\"{a}ngend. Graph (c) ist nicht zusammenh\"{a}ngend. Die 
gr\"{�}sste Komponente besteht aus den  Knoten $A$, $B$, $C$ und $E$.
\item[Aufgabe~\ref{exunter}] Graphen (a) und (c) sind Teilgraphen (Ein Teilgraph muss
  \emph{nicht} notwendig zusammenh\"{a}ngend sein). Graph (b)
  enth\"{a}lt eine Kante $\{A,C\}$ die im urspr\"{u}nglichen Graphen
  nicht existiert. 
\item[Aufgabe~\ref{excompletesubgraph}] 
Ein kompletter Graph $K_n$ hat $n(n-1)/2$ Kanten. F\"{u}r jede Kante 
wird die Anzahl der m\"{o}glichen Teilgraphen verdoppelt, da die Kante 
entweder im Teilgraphen vorkommen kann oder nicht. Somit hat $K_2$ zwei
verschiedene Teilgraphen, $K_3$ hat acht Teilgraphen, und $K_n$ hat $2^{n(n-1)/2}$ spannende
Teilgraphen.
\item[Aufgabe~\ref{extravel}]\hfill\linebreak
\begin{enumerate}
\item Z\"{u}rich---Singapur---Sydney (16589 km)
\item Z\"{u}rich---Moskau---Ulan Bator---Sydney (Fr. 3400.-)
\end{enumerate}
\end{description}

\section{Kapiteltest}\label{graphchaptest}

\begin{exercise}\label{test1adjacency}
Gegeben sei die Adjazenzmatrix f\"{u}r einen Graphen $G$.

\begin{displaymath}
A(G) = \left(
\begin{array}{ccccccc}
0 & 1 & 0 & 0 & 0 & 0 & 0 \\
1 & 0 & 0 & 1 & 0 & 0 & 0 \\
0 & 0 & 0 & 0 & 0 & 0 & 0 \\
0 & 1 & 0 & 0 & 1 & 0 & 0 \\
0 & 0 & 0 & 1 & 0 & 0 & 0 \\
0 & 0 & 0 & 0 & 0 & 0 & 1 \\
0 & 0 & 0 & 0 & 0 & 1 & 0 
\end{array}
\right)
\end{displaymath}

\begin{enumerate}
\item Zeichnen Sie den Graphen auf (verwenden Sie $A..G$ als
  Knotennamen).
\item Stellen Sie den Graphen in Mengennotation dar.
\item Aus welchen Komponenten besteht der Graph?
\end{enumerate}

\end{exercise}

\begin{exercise}\label{test1weights}
Gegeben sei der folgende Graph:

\scalebox{0.5}{\includegraphics{SubWegGewicht.pdf}}

\begin{enumerate}
\item Welcher Weg von $A$ nach $D$ weist die kleinste Summe von
  Kantengewichten auf?
\item Welcher geschlossene nichtleere Weg im Graphen weist die kleinste Summe von
  Kantengewichten auf?
\end{enumerate}

\end{exercise}

\section{L\"{o}sung zum Kapiteltest}
\begin{description}
\item[Aufgabe~\ref{test1adjacency}] \hfill \\[0cm]
\begin{enumerate}
\item 
\begin{figure}[H]
\centerline{
\scalebox{0.5}{\includegraphics{kapiteltest1.pdf}}
}
\end{figure}

\item $G = (V,E)$, $V = \{A, B, C, D, E\}$, $E=\{ \{A, B\}, \{B, D\},\{D, E\},\{F, G\}\}$.
\item Der Graph besteht aus drei Komponenten. Die Knoten $A$,$B$,$D$ und $E$ und die Kanten dazwischen bildet 
eine Komponente, Knoten $C$ ist die zweite Komponente, und $F$ und $G$ mit der Kante dazwischen bildet die dritte Komponente.
\end{enumerate}

\item[Aufgabe~\ref{test1weights}] \hfill \\[0cm]
\begin{enumerate}
\item $A, E, C, D$ hat eine Kantengewichtsumme von nur $7$.
\item $A, B, C, E, A$ hat eine Kantengewichtsumme von $13$.
\end{enumerate}
\end{description}

\chapter{B\"{a}ume}

\section{Was ist ein Baum?}

Als \textbf{Baum} bezeichnen wir einen Graphen, der
\emph{zusammenh\"{a}ngend} ist und \emph{keinen geschlossenen Weg}
enth\"{a}lt. Eine Menge von B\"{a}umen, die keine gemeinsamen Knoten
haben, bezeichnet man (anschaulicherweise) als \textbf{Wald}.

\begin{figure}[h!]
\begin{subfigure}[b]{0.35\textwidth}
\resizebox{!}{5cm}{\includegraphics{BaumBsp.pdf}}
\caption{}
\end{subfigure}
\begin{subfigure}[b]{0.6\textwidth}
\resizebox{!}{7cm}{\includegraphics{WaldBsp.pdf}}
\caption{}
\end{subfigure}
\caption{Ein Baum (a) und ein Wald (b)}
\end{figure}

\begin{exercise}\label{exbaum}
Welche der folgenden drei Graphen sind B\"{a}ume? Gestalten Sie die
anderen Graphen durch Hinzuf\"{u}gen bzw. Weglassen von geeigneten
Kanten so um, dass sie ebenfalls zu B\"{a}umen werden.

\begin{figure}[h!]
\begin{subfigure}[b]{0.3\textwidth}
\scalebox{0.4}{\includegraphics{BaumLoch.pdf}}
\caption{}
\end{subfigure}
\begin{subfigure}[b]{0.3\textwidth}
\scalebox{0.4}{\includegraphics{Baum.pdf}}
\caption{}
\end{subfigure}
\begin{subfigure}[b]{0.3\textwidth}
\scalebox{0.4}{\includegraphics{BaumKreis.pdf}}
\caption{}
\end{subfigure}
\caption{Welche dieser Graphen sind B\"{a}ume?}
\end{figure}

\end{exercise}

\newpage
\begin{exercise}\label{exvertnode}
Welche Aussage l\"{a}sst sich \"{u}ber die Anzahl Kanten eines Baums
mit $n$ Knoten machen?
\end{exercise}

\section{Spannb\"{a}ume}

Ein \textbf{Spannbaum} f\"{u}r einen zusammenh\"{a}ngenden Graphen
$G(V,E)$ ist ein Baum, der \emph{alle Knoten} $V$ des
Graphen und eine \emph{Teilmenge der Kanten} $E$ enth\"{a}lt. Jeder
zusammenh\"{a}ngende Graph hat somit einen oder mehrere Spannb\"{a}ume.

\begin{figure}[h!]
\begin{subfigure}[b]{0.24\textwidth}
\scalebox{0.4}{\includegraphics{Ungerichtet.pdf}}
\caption{}
\end{subfigure}
\begin{subfigure}[b]{0.24\textwidth}
\scalebox{0.4}{\includegraphics{UngerSpann1.pdf}}
\caption{}
\end{subfigure}
\begin{subfigure}[b]{0.24\textwidth}
\scalebox{0.4}{\includegraphics{UngerSpann2.pdf}}
\caption{}
\end{subfigure}
\begin{subfigure}[b]{0.24\textwidth}
\scalebox{0.4}{\includegraphics{UngerSpann3.pdf}}
\caption{}
\end{subfigure}
\caption{Ein Graph (a) und seine m\"{o}glichen Spannb\"{a}ume (b)--(d)}
\end{figure}

\begin{exercise}\label{exspan}
Z\"{a}hlen Sie alle m\"{o}glichen Spannb\"{a}ume dieses Graphen auf:\\*
\scalebox{0.5}{\includegraphics{Exspan.pdf}}
\end{exercise}

\newpage
\section{Minimale Spannb\"{a}ume}

Wenn wir jetzt wieder an das im Abschnitt~\ref{gewichtet}
eingef\"{u}hrte Konzept des \emph{kantengewichteten Graphen}
zur\"{u}ckdenken, k\"{o}nnen wir dieses auch auf B\"{a}ume
anwenden. Das \textbf{Gewicht} eines Baumes l\"{a}sst sich dann als
Summe der Kantengewichte des Baumes definieren.

\begin{exercise}\label{exgewicht}
Bestimmen Sie das Gewicht dieses Baumes:\\*
\nopagebreak\scalebox{0.4}{\includegraphics{Exgewicht.pdf}}
\end{exercise}

Somit k\"{o}nnen wir nun den zentralen Begriff dieses Textes
definieren: Ein Spannbaum $B$ eines Graphen $G$ ist ein
\textbf{minimaler Spannbaum} von $G$ wenn kein Spannbaum von $G$
existiert, dessen Gewicht kleiner als das Gewicht von $B$ ist.

\begin{exercise}\label{exspangewicht}
Finden Sie den minimalen Spannbaum dieser Graphen:

\begin{figure}[h!]
\begin{subfigure}[b]{0.3\textwidth}
\scalebox{0.6}{\includegraphics{Exspangewicht.pdf}}
\caption{}
\end{subfigure}
\begin{subfigure}[b]{0.3\textwidth}
\scalebox{0.6}{\includegraphics{Exspan_w1.pdf}}
\caption{}
\end{subfigure}
\begin{subfigure}[b]{0.3\textwidth}
\scalebox{0.6}{\includegraphics{Exspan_w2.pdf}}
\caption{}
\end{subfigure}
\end{figure}

(F\"{u}r Aufgaben (b) und (c) k\"{o}nnen Sie die Aufz\"{a}hlung der
Spannb\"{a}ume in Aufgabe~\ref{exspan} als Hilfsmittel verwenden.)
\end{exercise}

\newpage
\section{L\"{o}sungen}

\begin{description}
\item[Aufgabe~\ref{exbaum}] Der mittlere Graph (b) war bereits ein
  Baum.\\*
\begin{figure}[h!]
\setcounter{subfigure}{0}
\begin{subfigure}[b]{0.5\textwidth}
\scalebox{0.45}{\includegraphics{BaumLochFix.pdf}}
\caption{war nicht zusammenh\"{a}ngend}
\end{subfigure}
\stepcounter{subfigure}
\begin{subfigure}[b]{0.5\textwidth}
\scalebox{0.45}{\includegraphics{BaumKreisFix.pdf}}
\caption{hatte einen geschlossenen Pfad}
\end{subfigure}
\end{figure}

\item[Aufgabe~\ref{exvertnode}] 
Ein Baum mit $1$~Knoten kann keine Kanten haben. Wenn man nun zu 
einem bestehenden Baum einen weiteren Knoten hinzuf\"{u}gt, ben\"{o}tigt
man eine weitere Kante, um diesen Knoten mit dem Rest des Baums zu
verbinden. Folglich hat ein Baum mit $n$~Knoten \emph{mindestens}
$n-1$~Kanten.

In einem Baum $B(V,E)$ mit $n$~Knoten~$V$ und $n-1$~Kanten~$E$ besteht aber
definitionsgem\"{a}ss bereits ein Weg zwischen jedem Knotenpaar 
$\{A,B\}\in E$. Wenn man dann eine weitere Kante
$\{A,B\}$~hinzuf\"{u}gt, erh\"{a}lt man folglich einen geschlossenen
Weg, der $A$~und $B$ enth\"{a}lt, und der entstehende Graph ist kein
Baum mehr!

Somit sehen wir, dass ein Baum mit $n$~Knoten \emph{exakt}
$n-1$~Kanten enth\"{a}lt.
\item[Aufgabe~\ref{exspan}] Wenn man eine der vier Kanten $\{\{A,B\},
  \{A,D\}, \{B,C\}, \{C,D\}\}$ entfernt, erh\"{a}lt man je einen
  Spannbaum. Somit sind die vier m\"{o}glichen Spannb\"{a}ume:

\begin{itemize}
\item $\{\{A,B\}, \{A,D\}, \{B,C\}\}$
\item $\{\{A,B\}, \{A,D\}, \{C,D\}\}$
\item $\{\{A,B\}, \{B,C\}, \{C,D\}\}$
\item $\{\{A,D\}, \{B,C\}, \{C,D\}\}$
\end{itemize}

\item[Aufgabe~\ref{exgewicht}] 25
\item[Aufgabe~\ref{exspangewicht}] 
(a) Der minimale Spannbaum hat das
  Gewicht 8:\\*
\scalebox{0.5}{\includegraphics{Exspangewichtsol.pdf}}

(b) Es existieren zwei minimale Spannb\"{a}ume mit Gewicht 6:
\begin{itemize}
\item $\{\{A,B\}, \{A,D\}, \{B,C\}\}$
\item $\{\{A,B\}, \{A,D\}, \{C,D\}\}$
\end{itemize}

(c) Alle vier Spannb\"{a}ume sind minimal, mit Gewicht 4:
\begin{itemize}
\item $\{\{A,B\}, \{A,D\}, \{B,C\}\}$
\item $\{\{A,B\}, \{A,D\}, \{C,D\}\}$
\item $\{\{A,B\}, \{B,C\}, \{C,D\}\}$
\item $\{\{A,D\}, \{B,C\}, \{C,D\}\}$
\end{itemize}

Wir sehen, dass ein Graph ohne weiteres mehrere minimale
Spannb\"{a}ume haben kann. Wenn alle Kantengewichte identisch 
sind, sind \emph{alle} Spannb\"{a}ume minimal!
\end{description}

\newpage
\section{Kapiteltest}
\begin{exercise}\label{kapspan}
Gegeben sei folgender Graph $G$:

\scalebox{0.5}{\includegraphics{KapiteltestSubgraph.pdf}}\\*
Welche der folgenden Graphen sind B\"{a}ume, welche sind Spannb\"{a}ume von $G$?

\begin{figure}[h!]
\setcounter{subfigure}{0}
\begin{subfigure}[b]{0.3\textwidth}
\scalebox{0.5}{\includegraphics{KapiteltestSubgraph1.pdf}}
\caption{}
\end{subfigure}
\begin{subfigure}[b]{0.3\textwidth}
\scalebox{0.5}{\includegraphics{KapiteltestSubgraph2.pdf}}
\caption{}
\end{subfigure}
\begin{subfigure}[b]{0.3\textwidth}
\scalebox{0.5}{\includegraphics{KapiteltestSubgraph3.pdf}}
\caption{}
\end{subfigure}
\end{figure}
\end{exercise}
\begin{exercise}\label{kapspanmin}
Zeichne alle minimalen Spannb\"{a}ume dieses Graphen auf:

\scalebox{0.5}{\includegraphics{KapiteltestMST.pdf}}
\end{exercise}

\section{L\"{o}sungen zum Kapiteltest}
\begin{description}
\item[Aufgabe~\ref{kapspan}] (a)~ist ein Baum, aber \emph{kein}
  Spannbaum von $G$, weil die Kante~$\{B,E\}$ nicht in $G$ enthalten
  ist. (b)~ist ein Spannbaum von $G$. (c)~ist kein
  Baum, weil ein geschlossener Weg existiert.
\item[Aufgabe~\ref{kapspanmin}] \hfill\\*

\begin{figure}[h!]
\setcounter{subfigure}{0}
\begin{subfigure}[b]{0.5\textwidth}
\scalebox{0.5}{\includegraphics{KapiteltestMST_s1.pdf}}
\caption{}
\end{subfigure}
\begin{subfigure}[b]{0.5\textwidth}
\scalebox{0.5}{\includegraphics{KapiteltestMST_s2.pdf}}
\caption{}
\end{subfigure}
\end{figure}
\end{description}

\chapter{Algorithmen zur Bestimmung von minimalen Spannb\"{a}umen}

Wie findet man denn nun einen minimalen Spannbaum in einem
Graphen, der etwas gr\"{o}sser ist? In diesem Kapitel werden wir zwei
verschiedene Algorithmen kennenlernen, die einen minimalen Spannbaum
in einer Folge von \"{u}bersichtlichen Schritten bestimmen. Beide
dieser Algorithmen geh\"{o}ren der Familie der \emph{gierigen}
Algorithmen an.

\section{Gierige Algorithmen}

Ein Algorithmus wird \textbf{gierig} (bzw. das Englische Aequivalent
\textbf{greedy}) genannt, wenn er in jedem Schritt einen \emph{lokal
  optimalen} Folgezustand w\"{a}hlt, das heisst, den Zustand, der ohne
weitere Vorausplanung am besten aussieht. In Reiseproblem, das in
Aufgabe~\ref{extravel} besprochen wurde, w\"{u}rde ein gieriger
Algorithmus z.B. in jedem Schritt den k\"{u}rzesten oder den
billigsten Flug aus der gegenw\"{a}rtigen Stadt in eine noch nicht
besuchte Stadt w\"{a}hlen.

Gierige Algorithmen sind oft einfach zu verstehen und schnell, aber
sie l\"{o}sen viele Probleme nicht optimal: In unserem Beispiel
w\"{u}rde der Algorithmus zwar den billigsten Weg finden, aber im
allgemeinen Fall ist das nicht garantiert, und der Algorithmus
w\"{u}rde hier die falsche L\"{o}sung f\"{u}r den k\"{u}rzesten Weg
finden:

\begin{itemize}
\item Im ersten Schritt ist von Z\"{u}rich aus die Reise nach Moskau
  sowohl am k\"{u}rzesten als auch am billigsten.
\item Im n\"{a}chsten Schritt wollen wir nicht umkehren, also reisen
  wir nach Ulan Bator weiter.
\item Zuletzt fliegen wir nach Sydney weiter.
\end{itemize}

Folglich w\"{u}rde sowohl nach dem ``k\"{u}rzesten'' als auch nach dem
``billigsten'' Kriterium die Route Z\"{u}rich--Moskau--Ulan
Bator--Sydney gew\"{a}hlt, w\"{a}hrend in Wirklichkeit die Route
Z\"{u}rich--Singapur--Sydney k\"{u}rzer ist.

Gl\"{u}cklicherweise ist das Finden eines minimalen Spannbaums eines
der Probleme, bei denen gierige Algorithmen ein optimales Resultat
garantieren k\"{o}nnen. Wenn wir n\"{a}mlich einen minimalen Spannbaum 
f\"{u}r einen \emph{Teil} des Graphen nehmen und diesen um die 
\textbf{minimale Kante}\footnote{d.h. die Kante mit dem kleinsten Gewicht}
erweitern, die zu einem Knoten ausserhalb dieses Spannbaums f\"{u}hrt, 
k\"{o}nnen wir zeigen, dass wir das Resultat immer zu einem Spannbaum f\"{u}r
den \emph{ganzen} Graphen erweitern k\"{o}nnen:

\begin{proof}\label{minimalkante}
  Es sei $G(V,E)$ ein kantengewichteter
  zusammenh\"{a}ngender Graph. $U$ sei eine Teilmenge der Knoten $V$ und
  $e_{min}$ die Kante mit dem kleinsten Gewicht in der Kantenmenge
  $\{e=\{u,v\} \mid e\in E, u\in U, v\in (V\!\setminus\!U) \}$. Dann existiert ein
  minimaler Spannbaum von $G$, der $e_{min}$ enth\"{a}lt.

 \bigskip\noindent\textsc{Beweis:} Nehmen wir das Gegenteil an: $e_{min}$
liegt in \emph{keinem} minimalen Spannbaum von $G$. Wenn wir dann $B$,
einen beliebigen minimalen Spannbaum von $G$, nehmen und $e_{min}$ in $B$
einf\"{u}gen, erh\"{a}lt man einen geschlossenen Weg $W$ in $B$, der
$e_{min}$ enth\"{a}lt. Da $W$ Knoten sowohl aus $U$ als auch aus
$V\!\setminus\!U$ enth\"{a}lt, muss $W$ eine andere Kante $\{u', v'\}$
enthalten, in der $u' \in U, v'\in (V\!\setminus\!U)$ sind. Wenn man
dann $\{u',v'\}$ aus $B$ entfernt, bleibt der entstehende Baum $B'$ ein Spannbaum, und da nach
Definition das Gewicht von $e_{min}$ nicht gr\"{o}sser sein kann als
das von $\{u',v'\}$, kann auch das Gewicht von $B'$ nicht
gr\"{o}sser sein als das von $B$: $B'$ muss ebenfalls ein minimaler
Spannbaum sein, und da $B'$ $e_{min}$ enth\"{a}lt, ist die
Gegenannahme widerlegt.
\end{proof} 

Somit wissen wir, dass jeder gierige Algorithmus, der in jedem Schritt eine
solche minimale Kante ausw\"{a}hlt, zu einem minimalen Spannbaum f\"{u}hrt.

\newpage
\section{Der Algorithmus von Kruskal}

Der erste Algorithmus, den wir behandeln, wurde von dem
amerikanischen Mathematiker Joseph Kruskal entwickelt und erstmals
1956 publiziert. Die Grundidee in diesem Algorithmus ist, dass man in
jedem Schritt die minimale \emph{n\"{u}tzliche} Kante
ausw\"{a}hlt:

\algostep{DemoSmall.pdf}{Schauen wir uns diesen einfachen Graphen
an.}
\algostep{DemoSmallKruskal1.pdf}{Die minimale Kante ist $\{B, C\}$.}
\algostep{DemoSmallKruskal2.pdf}{Nachdem wir diese ausgew\"{a}hlt
  haben, ist die minimale verbleibende Kante $\{B,D\}$.}
\algostep{DemoSmallKruskal3.pdf}{Nun w\"{a}re die minimale
  verbleibende Kante ja eigentlich $\{C,D\}$. Wenn wir diese aber auch
ausw\"{a}hlen w\"{u}rden, h\"{a}tten wir aber einen geschlossenen Weg
$B$--$C$--$D$--$B$. Die Kante ist also nicht in unserem Sinn
n\"{u}tzlich, und wir d\"{u}rfen sie nicht verwenden.}
\algostep{DemoSmallKruskal4.pdf}{Somit verbleibt als minimale
  n\"{u}tzliche Kante $\{A,B\}$, und damit haben wir unseren minimalen
  Spannbaum vollendet.}

\newpage
Wenn wir uns das Vorgehen an einem etwas gr\"{o}sseren Beispiel
ansehen, sehen wir, dass zun\"{a}chst jeder Knoten
als separater Spannbaum betrachtet wird, und diese B\"{a}ume dann
sukzessive verbunden werden, bis ein einziger minimaler Spannbaum
\"ubrigbleibt:

\algostep{Demo.pdf}{Der Anfangszustand. Jeder Knoten bildet einen
  eigenen Spannbaum.}
\algostep{DemoKruskal1.pdf}{Schritt $1$: Eine der beiden Kanten mit Gewicht $5$
  (egal welche, da beide minimal sind) wird hinzugef\"{u}gt. $A$, $D$, und die verbindende
  Kante bilden somit einen Spannbaum, w\"{a}hrend die anderen Knoten
  immer noch eigene Spannb\"{a}ume bilden.}
\algostep{DemoKruskal2.pdf}{Schritt $2$: Die k\"{u}rzeste verbleibende
  Kante (die ebenfalls das Gewicht $5$ hat) wird hinzugef\"ugt.}
\algostep{DemoKruskal3.pdf}{Schritt $3$: Nachdem zwei weitere Kanten
  hinzugef\"ugt wurden, wird jetzt klar, dass die Kante $\{B,C\}$ nie
  mehr n\"{u}tzlich sein wird, weil sie mit $\{B,E\}$ und $\{C,E\}$
  einen geschlossenen Weg bilden w\"{u}rde. Wir haben nun drei
  Spannb\"{a}ume, die die Knoten $\{A, D, F\}$, $\{B, C, E\}$ und
  $\{G\}$ verkn\"{u}pfen.}

\newpage
\begin{exercise}\label{exkruskal}
F\"{u}hren Sie den Algorithmus von Kruskal im vorstehenden Beispiel zu
Ende, bis der minimale Spannbaum konstruiert ist.
\end{exercise}

Mathematisch etwas pr\"{a}ziser ausgedr\"{u}ckt, konstruiert der
Algorithmus von Kruskal eine minimalen Spannbaum f\"{u}r den 
Graphen~$G(V,E)$, indem ein Wald von minimalen B\"{a}umen~$W(V, E')$
sukzessive verbunden wird, bis ein einziger minimaler Spannbaum
\"ubrigbleibt:

\begin{enumerate}
\item Der Anfangszustand des Waldes besteht aus einem Baum pro Knoten:
  $W \gets (V, \{\})$ (Beachten Sie, dass ein
  einzelner Knoten bereits einen Baum darstellt)
\item So lange es Kanten $K$ mit $K\subset (E\setminus{E'})$ gibt, die in $W$ noch nicht
  verwendet wurden, und die mit den verwendeten Kanten $E'$ keinen
  geschlossenen Weg bilden:
\begin{enumerate}
\item W\"{a}hlen Sie die Kante $e_i\in K$ mit dem kleinsten
  Kantengewicht aus.
\item F\"{u}gen Sie diese Kante zu $W$ hinzu: $E' \gets E'+\{e_i\}$.
\end{enumerate}
\item Wenn jede verbleibende Kante in $E\setminus{E'}$ mit den
  verwendeten Kanten einen geschlossenen Weg bilden w\"{u}rde, ist der
  minimale Spannbaum $B(V,E')$ gefunden und der Algorithmus ist beendet.
\end{enumerate}

FAKULTATIV: Der Beweis des Algorithmus von Kruskal ist etwas
schwieriger als die anderen Beweise, die wir in diesem Text diskutiert
haben. Sie k\"{o}nnen ihn im Abschnitt~\ref{kruskproof} finden.

\newpage
\section{Der Algorithmus von Prim}

Viele Optimierungsprobleme k\"{o}nnen auf viele verschiedene Arten
gel\"{o}st werden. Auch f\"{u}r das minimale Spannbaumproblem gibt es
viele Algorithmen. Wir wollen deshalb noch einen zweiten
Algorithmus vorstellen, um einen anderen Ansatz zu zeigen.

Dieser Algorithmus wurde bereits 1930 von dem Mathematiker
Vojt\v{e}ch Jarn\'ik entwickelt, geriet dann aber in Vergessenheit,
bis er 1957 von Robert C. Prim und 1959 von Edsger W. Dijkstra
unabh\"{a}ngig wiederentdeckt wurde (Aus diesem Grund ist der
Algorithmus auch unter anderen Namen, z.B. \emph{Prim-Dijkstra
  Algorithmus} oder \emph{Algorithmus von Jarnik, Prim, und Dijkstra}
bekannt).

Die Grundidee des Algorithmus von Prim ist, dass man, von einem
Ausgangspunkt ausgehend, in jedem Schritt die minimale Kante
hinzuf\"{u}gt, die den bestehenden Spannbaum um einen weiteren Knoten
erweitert:

\algostep{DemoSmallPrim0.pdf}{Beginnen wir nochmals mit dem schon bekannten
  Beispiel und w\"{a}hlen wir Knoten $A$ als Ausgangspunkt und
  trivialen Spannbaum.}
\algostep{DemoSmallPrim1.pdf}{Die einzige Kante, die diesen Spannbaum
  erweitert, ist $\{A, B\}$.}
\algostep{DemoSmallPrim2.pdf}{Nun haben wir zwei M\"{o}glichkeiten zur
  Erweiterung und w\"{a}hlen davon die minimale Kante $\{B,C\}$.}
\algostep{DemoSmallPrim3.pdf}{Erneut haben wir zwei M\"{o}glichkeiten
  zur Erweiterung, und diesmal ist die minimale Kante $\{B,D\}$.}
\algostep{DemoSmallPrim4.pdf}{Damit haben wir unseren minimalen
  Spannbaum vollendet.}

\newpage 
Mathematisch etwas pr\"{a}ziser ausgedr\"{u}ckt, konstruiert
der Algorithmus von Prim einen minimalen Spannbaum f\"{u}r den
zusammenh\"{a}ngenden Graphen $G(V,E)$ indem er einen bestehenden Baum
$B(V_B,E_B)$ sukzessive durch Hinzuf\"{u}gen minimaler Kanten
erweitert:

\begin{enumerate}
\item W\"{a}hlen Sie einen beliebigen Knoten $v_0\in V$ als
  Anfangszustand f\"{u}r $B$: $V_B \gets \{v_0\}$, $E_B \gets \{\}$.
\item Solange $V_B\neq V$:
\begin{enumerate}
\item \label{primsel} W\"{a}hlen Sie unter den Kanten $\{e = \{u,v\}\mid u\in V_B, v\in
  (V\!\setminus\!V_B)\}$ die Kante $e_i= \{u_i,v_i\}$ mit dem kleinsten
Kantengewicht aus.
\item F\"{u}gen Sie die Kante und ihren Endpunkt zu $B$ hinzu: $V_B \gets
  V_B+\{v_i\}$, $E_B \gets E_B+\{e_i\}$.
\end{enumerate}
\end{enumerate}

FAKULTATIV: Auch beim Algorithmus von Prim ist der Beweis etwas
kompliziert. Sie k\"{o}nnen ihn im Abschnitt~\ref{primproof} finden.

\newpage
Um das nochmals an unserem etwas gr\"{o}sseren Beispiel zu illustrieren:

\algostep{DemoPrim1.pdf}{Der Anfangszustand, $V_B = \{A\}$ (beliebige
  Wahl), $E_B = \{\}$. M\"{o}gliche Erweiterungskanten sind $\{A,B\}$
  und $\{A,D\}$; letztere hat das kleinere Kantengewicht.}
\algostep{DemoPrim2.pdf}{Schritt $1$: $\{A,D\}$ bzw. $D$ wurden
  hinzugef\"{u}gt. $V_B = \{A,D\}, E_B=\{\{A,D\}\}$. Nun ist $\{D,F\}$ die
  beste Erweiterungskante.}
\algostep{DemoPrim3.pdf}{Schritt $2$: $\{D,F\}$ bzw. $F$ wurden 
 hinzugef\"{u}gt. $V_B = \{A,D,F\}, E_B=\{\{A,D\}, \{D,F\}\}$. Nun kommt
 endlich $\{A,B\}$ als Erweiterungskante zum Zug.}

\begin{exercise}\label{exprim}
F\"{u}hren Sie den Algorithmus von Prim im vorstehenden Beispiel zu
Ende, bis der minimale Spannbaum konstruiert ist.
\end{exercise}

\newpage
\section{Der Algorithmus von Prim f\"{u}r eine Adjazenzmatrix}

Der Algorithmus von Prim ist auch besonders gut geeignet, auf Papier einen
minimalen Spannbaum f\"{u}r einen Graphen in
Adjazenzmatrix-Darstellung zu konstruieren. Probieren wir das gleich
nochmals mit dem vorherigen Beispiel aus.

\vskip 5mm\matrixstep{
\begin{tabular}{|l|r|r|r|r|r|r|r|}\hline
   &   &   &   &   &   &   &   \\ \hline
   & A & B & C & D & E & F & G \\ \hline
A &   & 7 &   & 5 &   &    &    \\
B & 7 &   & 8 & 9 & 7 &   &   \\
C &   & 8 &   &   & 5 &   &   \\
D & 5 & 9 &   &   &15 & 6 &   \\
E &   & 7 & 5 &15 &   & 8 & 9 \\
F &   &   &   & 6 & 8 &   &11 \\
G &   &   &   &   & 9 &11 & \\ 
\hline
\end{tabular}}{
Konstruieren Sie die Adjazenzmatrix f\"{u}r ihren Graphen, indem Sie
f\"{u}r jede Kante das Kantengewicht in die entsprechenden Zellen
eintragen (Da die Matrix symmetrisch ist, w\"{a}re es der Einfachheit
halber auch m\"{o}glich, dies z.B. nur f\"{u}r die linke untere
H\"{a}lfte der Matrix zu tun).
}
\matrixstep{
\begin{tabular}{|l|r|r|r|r|r|r|r|}\hline
   & \ucl &   &   &   &   &   &   \\ \hline
   & A & B & C & D & E & F & G \\ \hline
\ign A &   & 7 &   & 5 &   &    &    \\
B & 7 &   & 8 & 9 & 7 &   &   \\
C &   & 8 &   &   & 5 &   &   \\
D & 5 & 9 &   &   &15 & 6 &   \\
E &   & 7 & 5 &15 &   & 8 & 9 \\
F &   &   &   & 6 & 8 &   &11 \\
G &   &   &   &   & 9 &11 & \\ 
\hline
\end{tabular}}{
W\"{a}hlen Sie dann einen beliebigen Knoten, z.B. $A$ aus. Markieren
Sie die Spalte dieses Knotens, um sich zu merken, dass $A$
jetzt zum Spannbaum geh\"{o}rt, und schraffieren Sie die Zeile des Knotens, um
anzuzeigen, dass keine Verbindungen nach $A$ mehr gemacht werden m\"{u}ssen.
}
\matrixstep{
\begin{tabular}{|l|r|r|r|r|r|r|r|}\hline
   & \ucl &   &   &   &   &   &   \\ \hline
   & A & B & C & D & E & F & G \\ \hline
\ign A &   & 7 &   & 5 &   &    &    \\
B & \byp{7} &   & 8 & 9 & 7 &   &   \\
C &   & 8 &   &   & 5 &   &   \\
D & \sel{5} & 9 &   &   &15 & 6 &   \\
E &   & 7 & 5 &15 &   & 8 & 9 \\
F &   &   &   & 6 & 8 &   &11 \\
G &   &   &   &   & 9 &11 & \\ 
\hline
\end{tabular}}{
Suchen Sie nun in allen markierten Spalten nach dem kleinsten
Gewicht. Sie finden die Kante $\{A,D\}$.
}
\matrixstep{
\begin{tabular}{|l|r|r|r|r|r|r|r|}\hline
   & \ucl &   &   & \ucl  &   &   &   \\ \hline
   & A & B & C & D & E & F & G \\ \hline
\ign A &   & 7 &   & \usd{5} &   &    &    \\
B & 7 &   & 8 & 9 & 7 &   &   \\
C &   & 8 &   &   & 5 &   &   \\
\ign D & \usd{5} & 9 &   &   &15 & 6 &   \\
E &   & 7 & 5 &15 &   & 8 & 9 \\
F &   &   &   & 6 & 8 &   &11 \\
G &   &   &   &   & 9 &11 & \\ 
\hline
\end{tabular}}{
Merken Sie sich diese Kante, markieren Sie die Spalte des neu
verbundenen Knotens $D$ und schraffieren Sie dessen Zeile.
}
\matrixstep{
\begin{tabular}{|l|r|r|r|r|r|r|r|}\hline
   & \ucl &   &   & \ucl  &   &   &   \\ \hline
   & A & B & C & D & E & F & G \\ \hline
\ign A &   & 7 &   & \usd{5} &   &    &    \\
B & \byp{7} &   & 8 & \byp{9} & 7 &   &   \\
C &   & 8 &   &   & 5 &   &   \\
\ign D & \usd{5} & 9 &   &   &15 & 6 &   \\
E &   & 7 & 5 &\byp{15} &   & 8 & 9 \\
F &   &   &   & \sel{6} & 8 &   &11 \\
G &   &   &   &   & 9 &11 & \\ 
\hline
\end{tabular}}{
Suchen Sie erneut in allen markierten Spalten nach dem kleinsten
Gewicht. Sie finden die Kante $\{D,F\}$.
}

\pagebreak
\begin{exercise}\label{exprimadj}
F\"{u}hren Sie auch diese Form des Algorithmus zu
Ende, bis der minimale Spannbaum konstruiert ist.
\end{exercise}

\section{Schlussbetrachtung: Wer Gewinnt?}

Wenn mehrere verschiedene Algorithmen zur L\"{o}sung eines Problems
zur Wahl stehen, ist es eine naheliegende Frage, welcher davon denn
der beste ist. Leider w\"{u}rde eine umfassende Erforschung dieser
Frage den Rahmen dieses Textes bei weitem sprengen, deshalb soll nur
kurz erw\"{a}hnt sein:

\begin{itemize}
\item Da alle Algorithmen hier optimale L\"{o}sungen finden,
  unterscheidet sich die Qualit\"{a}t der Resultate nicht.
\item Wenn man diese Algorithmen auf einem Computer programmiert,
  h\"{a}ngt deren Laufzeit- und Speichereffizienz enorm von den
  verwendeten Datenstrukturen ab, weil die Auswahl- und Testschritte
  (z.B. auf geschlossene Wege in Kruskal's Algorithmus, auf
  Mitgliedschaft von Knoten in Teilgraphen in Prim's Algorithmus)
  entscheidend auf eine effiziente Datenrepr\"{a}sentation angewiesen sind.
\item In einer Studie von mehreren Implementierungen von Kruskal's und
  Prim's Algorithmus ist der Informatikpionier Donald
  E. Knuth\footnote{Donald E. Knuth, \emph{The Stanford GraphBase},
    Seiten 460--497} zum Schluss
  gelangt, dass sich die Effizienz der beiden Algorithmen nicht
  wesentlich voneinander unterscheidet.
\end{itemize}

\newpage
\begin{exercise}\label{exfinal1}
Berechnen Sie die minimalen Spannb\"{a}ume f\"{u}r den folgenden
Graphen.

\scalebox{0.6}{\includegraphics{FinalExercise1.pdf}}
\end{exercise}
\begin{exercise}\label{exfinal2}
Berechnen Sie den minimalen Spannbaum f\"{u}r den folgenden Graphen.

\scalebox{0.6}{\includegraphics{FinalExercise2.pdf}}
\end{exercise}

\newpage
\section{L\"{o}sungen}

\begin{description}
\item[Aufgaben~\ref{exkruskal} und~\ref{exprim}] konstruieren letztendlich beide
den gleichen minimalen Spannbaum:

\scalebox{0.5}{\includegraphics{DemoResult.pdf}}
\item[Aufgabe~\ref{exprimadj}] konstruiert den gleichen Spannbaum, als
  Adjazenzmatrix:

\begin{tabular}{|l|r|r|r|r|r|r|r|}\hline
   & \ucl & \ucl  & \ucl & \ucl & \ucl & \ucl   & \ucl \\ \hline
   & A & B & C & D & E & F & G \\ \hline
\ign A &   & \usd{7} &   & \usd{5} &   &    &    \\
\ign B & \usd{7} &   & 8 & 9 & \usd{7} &   &   \\
\ign C &   & 8 &   &   & \usd{5} &   &   \\
\ign D & \usd{5} & 9 &   &   &15 & \usd{6} &   \\
\ign E &   & \usd{7} & \usd{5} &15 &   & 8 & \usd{9} \\
\ign F &   &   &   & \usd{6} & 8 &   &11 \\
\ign G &   &   &   &   & \usd{9} &11 & \\ 
\hline
\end{tabular}

\item[Aufgabe~\ref{exfinal1}]Es gibt zwei verschiedene minimale
  Spannb\"{a}ume, da Knoten $F$ mit zwei Kanten mit identischem
  Gewicht erreicht werden kann.\\*

\scalebox{0.5}{\includegraphics{Final1_1.pdf}}
\scalebox{0.5}{\includegraphics{Final1_2.pdf}}
\pagebreak
\item[Aufgabe~\ref{exfinal2}]\hfill\\*

\scalebox{0.6}{\includegraphics{Final2.pdf}}
\end{description}
\pagebreak

\section{Kapiteltest}

\begin{exercise}
Sie haben den Auftrag erhalten, ein Glasfaser-Netzwerk zu entwerfen,
das die gr\"{o}ssten Schweizer St\"{a}dte verbindet
(Elektrizit\"{a}t haben wir ja inzwischen alle). Gehen Sie dabei von
der Annahme aus, dass Sie die Glasfaser zwischen zwei beliebigen St\"{a}dten in
Luftlinie verlegen k\"{o}nnen.\\

\begin{tabular}{|l|c|r@{ / }r|}\hline
  Stadt & Abk\"{u}rzung & \multicolumn{2}{c|}{Koordinaten} \\
  \hline
  Z\"{u}rich & ZRH & 683248 & 248161 \\
  Genf &         GVA & 500532 & 117325 \\
  Basel &        BSL & 611587 & 267423 \\
  Lausanne & LAU & 538200 & 152026 \\
  Bern &         BRN & 600000 & 200000 \\
  Winterthur & WIN & 698805 & 261852 \\
  Luzern &     LZN & 665450 &  211356 \\
  St. Gallen & QGL & 746265 & 254310 \\
  Lugano &    LUG & 718030 & 96560 \\
  Biel &          BIE  & 585481 & 220742 \\
  \hline
  Thun &       THU & 614620 & 178664 \\
  K\"{o}niz & CHT & 598221 & 197101 \\
  La Chaux-de-Fonds & LCF & 553419 & 216894 \\
  Schaffhausen & SCH & 689677 & 283948 \\
  Freiburg & FRB & 578943 & 183921 \\
  Chur & CHR & 759742 & 190895 \\
  Neuenburg & QNC & 561352 & 204483 \\
  Vernier & VNR & 496673 & 117390 \\
  Uster & USR & 696755 & 245077 \\
  Sitten & SIR & 594446 & 120213 \\
  \hline
\end{tabular}\\

Auf der n\"achsten Seite finden Sie eine Distanzentabelle, die Ihnen
sicher gute Dienste leisten wird. 

\begin{enumerate}
\item Bestimmen Sie mit Hilfe eines der
beiden Algorithmen, die Sie nun kennen, den minimalen Spannbaum
f\"{u}r entweder die ersten 10 der St\"{a}dte, oder (fakultativ)
f\"{u}r alle 20 St\"{a}dte (Es ist nicht n\"{o}tig, das Resultat
aufzuzeichnen, eine Mengendarstellung der Verbindungen gen\"{u}gt).
\item Wieviel km Glasfaserkabel m\"{u}ssen Sie bestellen?
\end{enumerate}

\end{exercise}

\newpage
\pagestyle{empty}
\begin{sideways}
\footnotesize
\begin{tabular}{|l|r|r|r|r|r|r|r|r|r|r||r|r|r|r|r|r|r|r|r|r|}\hline
 & ZRH & GVA & BSL & LAU & BRN & WIN & LZN & QGL & LUG & BIE & THU & CHT & LCF & SCH & FRB & CHR & QNC & VNR & USR & SIR \\ \hline
ZRH & & 225 &  74 & 174 &  96 &  21 &  41 &  63 & 156 & 102 &  98 &  99 & 134 &  36 & 123 &  96 & 129 & 228 &  14 & 156 \\
GVA & 225 & & 187 &  51 & 129 & 245 & 190 & 281 & 218 & 134 & 130 & 126 & 113 & 252 & 103 & 269 & 106 &   4 & 234 &  94 \\
BSL &  74 & 187 & & 137 &  68 &  87 &  78 & 135 & 201 &  53 &  89 &  72 &  77 &  80 &  90 & 167 &  81 & 189 &  88 & 148 \\
LAU & 174 &  51 & 137 & &  78 & 195 & 140 & 232 & 188 &  83 &  81 &  75 &  67 & 201 &  52 & 225 &  57 &  54 & 184 &  65 \\
BRN &  96 & 129 &  68 &  78 & & 117 &  66 & 156 & 157 &  25 &  26 &   3 &  50 & 123 &  26 & 160 &  39 & 132 & 107 &  80 \\
WIN &  21 & 245 &  87 & 195 & 117 & &  61 &  48 & 166 & 121 & 118 & 120 & 152 &  24 & 143 &  94 & 149 & 248 &  17 & 176 \\
LZN &  41 & 190 &  78 & 140 &  66 &  61 & &  92 & 126 &  81 &  60 &  69 & 112 &  77 &  91 &  96 & 104 & 193 &  46 & 116 \\
QGL &  63 & 281 & 135 & 232 & 156 &  48 &  92 & & 160 & 164 & 152 & 159 & 196 &  64 & 182 &  65 & 192 & 285 &  50 & 203 \\
LUG & 156 & 218 & 201 & 188 & 157 & 166 & 126 & 160 & & 182 & 132 & 156 & 204 & 190 & 164 & 103 & 190 & 222 & 150 & 126 \\
BIE & 102 & 134 &  53 &  83 &  25 & 121 &  81 & 164 & 182 & &  51 &
27 &  32 & 122 &  37 & 177 &  29 & 136 & 114 & 101 \\ \hline
THU &  98 & 130 &  89 &  81 &  26 & 118 &  60 & 152 & 132 &  51 & &  25 &  72 & 129 &  36 & 146 &  59 & 133 & 106 &  62 \\
CHT &  99 & 126 &  72 &  75 &   3 & 120 &  69 & 159 & 156 &  27 &  25 & &  49 & 126 &  23 & 162 &  38 & 129 & 110 &  77 \\
LCF & 134 & 113 &  77 &  67 &  50 & 152 & 112 & 196 & 204 &  32 &  72 &  49 & & 152 &  42 & 208 &  15 & 115 & 146 & 105 \\
SCH &  36 & 252 &  80 & 201 & 123 &  24 &  77 &  64 & 190 & 122 & 129 & 126 & 152 & & 149 & 116 & 151 & 255 &  40 & 189 \\
FRB & 123 & 103 &  90 &  52 &  26 & 143 &  91 & 182 & 164 &  37 &  36 &  23 &  42 & 149 & & 181 &  27 & 106 & 133 &  66 \\
CHR &  96 & 269 & 167 & 225 & 160 &  94 &  96 &  65 & 103 & 177 & 146 & 162 & 208 & 116 & 181 & & 199 & 273 &  83 & 180 \\
QNC & 129 & 106 &  81 &  57 &  39 & 149 & 104 & 192 & 190 &  29 &  59 &  38 &  15 & 151 &  27 & 199 & & 108 & 141 &  91 \\
VNR & 228 &   4 & 189 &  54 & 132 & 248 & 193 & 285 & 222 & 136 & 133 & 129 & 115 & 255 & 106 & 273 & 108 & & 237 &  98 \\
USR &  14 & 234 &  88 & 184 & 107 &  17 &  46 &  50 & 150 & 114 & 106 & 110 & 146 &  40 & 133 &  83 & 141 & 237 & & 161 \\
SIR & 156 &  94 & 148 &  65 &  80 & 176 & 116 & 203 & 126 & 101 &  62 &  77 & 105 & 189 &  66 & 180 &  91 &  98 & 161 & \\
\hline\end{tabular}
\end{sideways}

\newpage
\section{L\"{o}sungen zum Kapiteltest}

Die gr\"{o}ssten 10 St\"{a}dte k\"{o}nnen mit 509km Glasfaserkabel
verbunden werden:

\scalebox{0.45}{\includegraphics{Cities10.pdf}}

\newpage
Die gr\"{o}ssten 20 St\"{a}dte k\"{o}nnen mit 712km Glasfaserkabel
verbunden werden:

\begin{sideways}
\scalebox{0.35}{\includegraphics{Cities.pdf}}
\end{sideways}

\appendix

\chapter{Beweise}

\section{Algorithmus von Kruskal}

\begin{proof}\label{kruskproof}Der Algorithmus von Kruskal konstruiert
  einen minimalen Spannbaum.

 \bigskip\noindent\textsc{Beweis:} Der Beweis besteht aus zwei Teilen. Zuerst zeigen wir, dass der Algorithmus von Kruskal einen Spannbaum produziert, danach, dass dieser Spannbaum minimal ist.
 
 \begin{enumerate}
 \item \textbf{Spannbaum:} Angenommen, Graph $W=(V,E')$ ist das
   Resultat einer Ausf\"{u}hrung des Algorithmus von Kruskal auf dem
   Graphen $G=(V,E)$. Dann hat der Graph $W$ keinen geschlossenen Weg,
   da die Kanten im Algorithmus per Definition so gew\"{a}hlt werden,
   dass kein geschlossener Weg entsteht. Wenn wir beweisen k\"{o}nnen,
   dass Graph $W$ auch zusammenh\"{a}ngend ist, folgt daraus, dass $W$
   ein Spannbaum ist. Um dies zu beweisen, nehmen wir an, dass $W$ aus
   zwei getrennten Komponenten $W_1$ und $W_2$ besteht. Da $G$
   zusammenh\"{a}ngend ist, existiert in $G$ mindestens eine Kante von
   einem Knoten aus $W_1$ zu einem Knoten aus $W_2$. Dies heisst
   jedoch, dass der Algorithmus noch nicht fertig ist, da eine solche
   Kante noch nicht in $W$ liegt, und auch mit den Kanten von $E'$
   keinen geschlossenen Weg bildet.
\item \textbf{Minimalit\"{a}t:} Dieser Beweis folgt durch
  Induktion. Wir betrachten jeden Schritt $k$ zwischen $0$ und der
  Anzahl Kanten des Spannbaums. Wir zeigen dann, dass der Graph $W_k=(V,E')$, den wir nach
  dem $k$-ten Schritt des Algorithmus erhalten, ein Teilgraph eines
  minimalen Spannbaums ist.
  Da wir oben bewiesen haben, dass wir nach dem letzten
  Schritt des Algorithmus einen Spannbaum erhalten, w\"{u}rde dies
  bedeuten, dass der erhaltene Spannbaum minimal ist. Den
  Induktionsanfang setzen wir bei $W_0=(V,\{\})$. Dieser Graph ist
  offensichtlich ein Teilgraph eines minimalen Spannbaums, da er
  mit dem minimalen Spannbaum knotengleich ist, und die Kantenmenge
  von $W_0$ leer ist, und somit Teilmenge jeder beliebigen Kantenmenge. F\"{u}r den
  Induktionsschritt nehmen wir an, dass $W_k=(V,E'_k)$ ein Teilgraph des 
  minimalen Spannbaum $T_k$ ist. Schauen wir jetzt an,
  was geschieht, wenn wir im $k+1$-ten Schritt $W_k$ um die Kante $e_{k+1}$ zu
  $W_{k+1}=(V,E'_k \cup \{e_{k+1}\})$ erweitern. Falls $e_{k+1}$ in $T_k$ vorkommt,
  l\"{a}sst sich auch $W_{k+1}$ zu $T_k$ erweitern, und wir sind
  fertig mit dem Induktionsschritt. Falls $e_{k+1}$ nicht in $T_k$ vorkommt,
  hat $T_k \cup \{e_{k+1}\}$ einen geschlossenen Weg. In diesem
  geschlossenen Weg existiert eine Kante $f \neq e_{k+1}$, welche in $W_k$
  nicht vorkommt (Ansonsten w\"{u}rde $e_{k+1}$ mit $W_k$ einen
  geschlossenen Weg formen). Da $e_{k+1}$ und nicht $f$ im $k+1$-ten Schritt
  des Algorithmus ausgew\"{a}hlt wurde, muss das Kantengewicht von $e_{k+1}$
  kleiner oder gleich dem Kantengewicht von $f$ sein. $T_k \cup \{e_{k+1}\}
  \setminus \{f\}$ ist somit ein Spannbaum mit einem h\"{o}chstens so
  hohen Kantengewicht wie $T_k$. Also muss $T_k \cup \{e_{k+1}\} \setminus
  \{f\}$ auch ein minimaler Spannbaum sein. Da $W_{k+1} = W_k \cup \{e_{k+1}\}
  $ ein Teilgraph von $T_k \cup \{e_{k+1}\}
  \setminus \{f\}$ ist, haben wir die Aussage bewiesen.
\end{enumerate}

 \end{proof}

\newpage
\section{Algorithmus von Prim}

\begin{proof}\label{primproof}Der Algorithmus von Prim konstruiert
  einen minimalen Spannbaum.

 \bigskip\noindent\textsc{Beweis:} Es ist leicht zu sehen, dass der
 Algorithmus einen Spannbaum $T$ konstruiert: In jedem Schritt wird
 $V_B$ um genau einen Knoten und $E_B$ um genau eine Kante
 erweitert, die den Knoten mit dem bestehenden Graphen verbindet. Der
 Algorithmus konstruiert also einen zusammenh\"{a}ngenden Graphen mit $V_B =
 V$ und $\textrm{(Anzahl Kanten)} = \textrm{(Anzahl Knoten)}-1$, was
 der Definition eines Spannbaums entspricht.

 Es bleibt noch zu zeigen, dass $T$ minimal ist: Wir wissen, dass
 mindestens ein minimaler Spannbaum, nennen wir ihn $T'$, existieren
 muss. Wenn $T' \neq T$, w\"{a}hlen wir die erste Kante $e_k = \{u,v\}$
 aus Schritt~2.1 des Algorithmus aus, die nicht in $T'$
 enthalten ist. Es muss also in $T'$ einen anderen Weg von $u$ nach
 $v$ geben, und daraus w\"{a}hlen wir eine Kante aus, die die bereits
 in vorherigen Schritten gew\"{a}hlten Knoten $V_B^{k-1}$ mit den noch
 nicht gew\"{a}hlten Knoten verbindet: $e' = \{u',v'\}, u'
 \in V_B^{k-1}, v' \in (V\!\setminus\!V_B^{k-1})$. Das Gewicht von
 $e'$ kann nicht kleiner sein als das von $e_k$, weil sonst im Schritt
 $k$ die Kante $e'$ gew\"{a}hlt worden w\"{a}re. Wenn wir also $e'$
 durch $e_k$ ersetzen, wird das Gewicht des resultierenden Baums nicht
 gr\"{o}sser, und diese Ersetzungen k\"{o}nnen wiederholt werden, bis
 $T' = T$.

\end{proof}

\end{document}
